\documentclass[a4paper,oneside,12pt]{article}

\usepackage{./custom}

\title{Projet DJ'Oz \\ \small{FSAB1402}}
\author{Antoine Legat \\(4776-1300) \and John de Wasseige \\(5224-1300)}
\date{4 décembre 2014}

\newcommand{\fun}[1]{\texttt{#1}}
\newcommand{\bigO}[1]{$\mathcal{O(\text{$ #1 $})}$}

\begin{document}

\maketitle
% structure, décisions de conception, difficultés rencontrées, limitations et problèmes connus
% justifier les constructions non déclaratives 
% complexité des fonctions 
% décrire les extensions

\section{Structure du programme}

\section{Complexité des fonctions}

Les complexités des différentes sous-fonctions de \fun{Inteprete} (resp. \fun{Mix})
se trouvent dans le tableau \ref{tab:complexite_interprete} 
(resp. tableau \ref{tab:complexite_mix}).


\subsection{Interprete}

\begin{table}[h]
	\centering
	\begin{tabular}{|l|c|c|}
		\hline
		Fonction & Complexité & Commentaires  \\
		\hline \hline
		\fun{ToNote} & \bigO{n} & Here is a test commentaire.  \\
	       	\fun{GivesH}  & \bigO{n} & test  \\
		\fun{Etirer} & \bigO{n} & test  \\
		\fun{Bourdon} & \bigO{n} & test  \\
		\fun{Transpose} & \bigO{n} & test  \\
		\fun{GivesDureeTot} & \bigO{n} & test  \\
		\hline
	\end{tabular}
	\caption{Complexités des fonctions de \fun{Interprete}.}
	\label{tab:complexite_interprete}
\end{table}

\subsection{Mix}



\begin{table}[h]
	\centering
	\begin{tabular}{|l|c|c|}
		\hline
		Fonction & Complexité & Commentaires  \\
		\hline \hline
		\fun{MixVoix} & \bigO{n} & Here is a test commentaire.  \\
	       	\fun{MixSilence}  & \bigO{n} & test  \\
		\fun{MixEch} & \bigO{n} & test  \\
		\hline
		\fun{Merge} & \bigO{n} & test  \\
		\fun{MergeAux} & \bigO{n} & test  \\		
		\hline
		\fun{RepeteN} & \bigO{n} & test  \\
		\fun{RepeteD} & \bigO{n} & test  \\ 
		\fun{Clip} & \bigO{n} & test  \\
		\fun{Echo} & \bigO{n} & test  \\
		\fun{CalcFirstIntensity} & \bigO{n} & test  \\
		\fun{Fondu} & \bigO{n} & test  \\
		\fun{FonduE} & \bigO{n} & test  \\
		\fun{Couper} & \bigO{n} & test  \\
		\hline
	\end{tabular}
	\caption{Complexités des fonctions de \fun{Mix}.}
	\label{tab:complexite_mix}
\end{table}

\section{Extensions}

\subsection{Lissage}

\subsection{Instrument}


\end{document}
