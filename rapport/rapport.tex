\documentclass[a4paper,oneside,10pt]{article}

\usepackage{./custom}
\usepackage[utf8]{inputenc}
\usepackage{layout}
\usepackage{graphicx}
\usepackage{fullpage}
\usepackage{textcomp}

\title{Projet DJ'Oz \\ \small{FSAB1402}}
\author{Antoine Legat \\(4776-1300) \and John de Wasseige \\(5224-1300)}
\date{4 décembre 2014}

\newcommand{\fun}[1]{\texttt{#1}}
\newcommand{\bigO}[1]{$\mathcal{O(\text{$ #1 $})}$}
\newcommand{\bslash}{\texttt{\symbol{92}}}

\begin{document}

\maketitle
% structure, décisions de conception, difficultés rencontrées, limitations et problèmes connus
% justifier les constructions non déclaratives 
% complexité des fonctions 
% décrire les extensions

\section{Structure du programme}
\fun{code.oz} crée un fichier audio \fun{out.wav} à partir d'une partition d'un certain format \fun{example.dj.oz} par le biais des fonctions \fun{Mix} et \fun{Interprete}.

Pour une meilleure lisibilité du code, nous avons décidé d'insérer le code de ces deux fonctions (ainsi que leurs sous-fonctions auxiliaires) avec la commande \fun{\bslash insert 'fileName.oz'}. Ce choix, en plus de réduire l'indentation, fut extrêmement pratique pour implémenter et debugger nos fonctions. Il est à noter que cette commande fonctionne exactement comme le \fun{\bslash input\{fileName.tex\}} du \LaTeX.

Nous avons procédé de la même manière pour implémenter \fun{Interprete} et \fun{Mix}. La fonction-mère contient un grand \fun{case Partition\bslash Music} qui gère tous les cas d'input que le programme doit pouvoir gérer. Selon le pattern reconnu, la fonction-mère va alors appeler une sous-fonction auxiliaire qui gère cette possibilité en particulier. L'usage de ces sous-fonctions, en plus de rendre le code plus lisible, est indispensable puisque la plupart sont récursives.

Enfin, nous avons choisi comme fichier de test l'incontournable \textit{Concerning Hobbits} d'Howard Shore. En effet, étant donné la très prochaine sortie du dernier volet de la saga \textit{The Hobbit}, nous tenions à remettre ce chef d'oeuvre de la saga initiale au goût du jour !

\section{Décisions de conception}

\paragraph{Assert}
Appréciant la limpidité de la procédure \fun{Assert} définie dans \fun{Projet2014.oz}, nous avons recopié sa définition afin de pouvoir l'utiliser dans nos fonctions. Ce fut d'une grande aide pour débugger et cela rend également nos fonctions plus explicites si elles crashent à cause d'un mauvais input. Nous avons également utilisé de temps à autres un \fun{raise} lorsque c'était plus adapté.

\paragraph{Robustesse}
Nous avons tenu à rendre notre code plus robuste en convertissant l'input dès que c'était raisonnable et possible. Par exemple, pour \fun{RepeteD}, la durée doit être un \fun{float} positif, mais nous la convertissons si ce n'est pas le cas (p.ex. \fun{$\sim$ 42} sera converti en \fun{42.0}).

\paragraph{Eviter \fun{Append}}
Nous avons tenu à éviter l'usage de la fonction \fun{Append} car elle est de complexité linéaire \bigO{n}. Nous concaténons donc en utilisant le symbole |, ainsi qu'un \fun{Reverse} ou un \fun{Flatten} quand cela s'avère nécessaire.

\paragraph{Récursion terminale}
Nous avons veillé à ce que nos fonctions soient récursives terminales, soit par l'usage d'un accumulateur, soit (pour les listes) en utilisant judicieusement l'avantage de la programmation déclarative, c'est-à-dire en renvoyant quelque chose comme \fun{Résultat|\{Fonction Reste\}}.
	
\paragraph{Constructions non-déclaratives}
N'en ayant pas eu besoin, nous n'en avons pas utilisé.
% assert
% float int robustesse abs
% | et pas append + reverse
% Acc réc terminales
% utilisé au max api (reverse, map,...)

\section{Complexité des fonctions}

Les complexités des différentes sous-fonctions de \fun{Inteprete} (resp. \fun{Mix})
se trouvent dans le tableau \ref{tab:complexite_interprete} 
(resp. tableau \ref{tab:complexite_mix}).


\subsection{Interprete}

\begin{table}[h]
	\centering
	\begin{tabular}{|l|c|c|}
		\hline
		Fonction & Complexité & Commentaires  \\
		\hline \hline
		\fun{ToNote} & \bigO{n} & Here is a test commentaire.  \\
	       	\fun{GetHauteur}  & \bigO{n} & test  \\
		\fun{Etirer} & \bigO{n} & test  \\
		\fun{Bourdon} & \bigO{n} & test  \\
		\fun{Transpose} & \bigO{n} & test  \\
		\fun{Instrument} & \bigO{n} & test  \\
		\fun{GivesDureeTot} & \bigO{n} & test  \\
		\hline
	\end{tabular}
	\caption{Complexités des fonctions de \fun{Interprete}.}
	\label{tab:complexite_interprete}
\end{table}

\subsection{Mix}



\begin{table}[h]
	\centering
	\begin{tabular}{|l|c|c|}
		\hline
		Fonction & Complexité & Commentaires  \\
		\hline \hline
		\fun{MixVoix} & \bigO{n} & Here is a test commentaire.  \\
	       	\fun{MixSilence}  & \bigO{n} & test  \\
		\fun{MixEch} & \bigO{n} & test  \\
		\fun{UseWav} & \bigO{n} & test  \\
		\fun{GetNote} & \bigO{n} & test  \\
		\hline
		\fun{Merge} & \bigO{n} & test  \\
		\fun{MergeAux} & \bigO{n} & test  \\		
		\hline
		\fun{RepeteN} & \bigO{n} & test  \\
		\fun{RepeteD} & \bigO{n} & test  \\ 
		\fun{Clip} & \bigO{n} & test  \\
		\fun{Echo} & \bigO{n} & test  \\
		\fun{CalcFirstIntensity} & \bigO{n} & test  \\
		\fun{Fondu} & \bigO{n} & test  \\
		\fun{FonduE} & \bigO{n} & test  \\
		\fun{Couper} & \bigO{n} & test  \\
		\hline
	\end{tabular}
	\caption{Complexités des fonctions de \fun{Mix}.}
	\label{tab:complexite_mix}
\end{table}

\section{Extensions}

\subsection{Lissage}

\subsection{Instrument}


\end{document}
