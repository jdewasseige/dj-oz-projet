\documentclass[a4paper,oneside,10pt]{article}

\usepackage{./custom}
\usepackage[utf8]{inputenc}
\usepackage{layout}
\usepackage{graphicx}
\usepackage{fullpage}
\usepackage{textcomp}

\title{Projet DJ'Oz \\ \small{FSAB1402}}
\author{Antoine Legat \\(4776-1300) \and John de Wasseige \\(5224-1300)}
\date{4 décembre 2014}

\newcommand{\fun}[1]{\texttt{#1}}
\newcommand{\bigO}[1]{$\mathcal{O(\text{$ #1 $})}$}
\newcommand{\bslash}{\texttt{\symbol{92}}}

\begin{document}

\maketitle
% difficultés rencontrées, limitations et problèmes connus
% complexité des fonctions 
% décrire les extensions

\section{Structure du programme}
\fun{code.oz} crée un fichier audio \fun{out.wav} à partir d'une partition d'un certain format \fun{example.dj.oz} par le biais des fonctions \fun{Mix} et \fun{Interprete}.

Pour une meilleure lisibilité du code, nous avons décidé d'insérer le code de ces deux fonctions (ainsi que leurs sous-fonctions auxiliaires) avec la commande \fun{\bslash insert 'fileName.oz'}. Ce choix, en plus de réduire l'indentation, fut extrêmement pratique pour implémenter et debugger nos fonctions. Il est à noter que cette commande fonctionne exactement comme le \fun{\bslash input\{fileName.tex\}} du \LaTeX.

Nous avons procédé de la même manière pour implémenter \fun{Interprete} et \fun{Mix}. La fonction-mère contient un grand \fun{case Partition\bslash Music} qui gère tous les cas d'input que le programme doit pouvoir gérer. Selon le pattern reconnu, la fonction-mère va alors appeler une sous-fonction auxiliaire qui gère cette possibilité en particulier. L'usage de ces sous-fonctions, en plus de rendre le code plus lisible, est indispensable puisque la plupart sont récursives.

Enfin, nous avons choisi comme fichier de test l'incontournable \textit{Concerning Hobbits} d'Howard Shore. En effet, étant donné la très prochaine sortie du dernier volet de la saga \textit{The Hobbit}, nous tenions à remettre ce chef d'oeuvre de la saga initiale au goût du jour ! Pour écrire sa partition, nous avons notamment utilisé \fun{muet}, \fun{duree}, \fun{etirer}, \fun{bourdon}, \fun{transpose}.

\paragraph{Limitations}
Le plus grand problème est certainement la lenteur de l'implémentation Mozart 2. 
On compte en moyenne $50$ à $60$ secondes pour produire un fichiers \fun{wav} d'une 
durée d'environ $1$ minute.

\section{Décisions de conception}

\paragraph{Assert}
Appréciant la limpidité de la procédure \fun{Assert} définie dans \fun{Projet2014.oz}, nous avons recopié sa définition afin de pouvoir l'utiliser dans nos fonctions. Ce fut d'une grande aide pour débugger et cela rend également nos fonctions plus explicites si elles crashent à cause d'un mauvais input. Nous avons également utilisé de temps à autres un \fun{raise} lorsque c'était plus adapté.

\paragraph{Robustesse}
Nous avons tenu à rendre notre code plus robuste en convertissant l'input dès que c'était raisonnable et possible. Par exemple, pour \fun{RepeteD}, la durée doit être un \fun{float} positif, mais nous la convertissons si ce n'est pas le cas (p.ex. \fun{$\sim$ 42} sera converti en \fun{42.0}).

\paragraph{Eviter \fun{Append}}
Nous avons tenu à éviter l'usage de la fonction \fun{Append} car elle est de complexité linéaire \bigO{n}. Nous concaténons donc en utilisant le symbole |, aidé d'un \fun{Reverse} ou un \fun{Flatten} quand cela s'avère nécessaire.

\paragraph{Récursion terminale}
Nous avons veillé à ce que nos fonctions soient récursives terminales, soit par l'usage d'un accumulateur, soit (pour les listes) en utilisant judicieusement l'avantage de la programmation déclarative, c'est-à-dire en renvoyant quelque chose comme \fun{Résultat|\{Fonction Reste\}}.
	
\paragraph{Constructions non-déclaratives}
N'en ayant pas eu besoin, nous n'en avons pas utilisé.

\paragraph{Temps de calcul}
Nous avons ajouté à \fun{code.oz} deux lignes qui mesurent le temps que prend la machine à calculer le fichier audio et qui l'affichent.

\paragraph{\fun{Echo} buggé}
Nous n'avons pas eu le temps de debugger \fun{Echo} qui ne fonctionne pas pour une raison inconnue.

\section{Complexité des fonctions}

Les complexités des différentes sous-fonctions de \fun{Inteprete} (resp. \fun{Mix})
se trouvent dans le tableau \ref{tab:complexite_interprete} 
(resp. tableau \ref{tab:complexite_mix}).

On remarque que notre code a une complexité générale \emph{linéaire} car nous n'utilisons nulle part de boucle et nous n'appelons jamais deux fois la même fonction récursive dans une fonction.
\subsection{Interprete}

\begin{table}[h]
	\centering
	\begin{tabular}{|l|c|c|}
		\hline
		Fonction & Complexité & Commentaires  \\
		\hline \hline
		\fun{Inteprete} & \bigO{n} & Taille de la partition. \\ 
		\fun{ToNote} & \bigO{1} & Transforme une note en une note étendue. \\
	       	\fun{GetHauteur}  & \bigO{1} & Donne la hauteur d'une note.  \\
		\fun{Etirer} & \bigO{n} & Longueur de Voix  \\
		\fun{Bourdon} & \bigO{n} & Longueur de Voix \\
		\fun{Transpose} & \bigO{n} & Longueur de Voix  \\
		\fun{Instrument} & \bigO{n} & Longueur de Voix  \\
		\fun{GivesDureeTot} & \bigO{n} & Longueur de Voix  \\
		\hline
	\end{tabular}
	\caption{Complexités des fonctions de \fun{Interprete}.}
	\label{tab:complexite_interprete}
\end{table}

\subsection{Mix}



\begin{table}[h]
	\centering
	\begin{tabular}{|l|c|c|}
		\hline
		Fonction & Complexité & Commentaires  \\
		\hline \hline
		\fun{MixVoix} & \bigO{n} & Longueur de la musique  \\
	       	\fun{MixSilence}  & \bigO{n} & Longueur du vecteur audio \\
		\fun{MixEch} & \bigO{n} & Longueur du vecteur audio \\
		\fun{UseWav} & \bigO{1} & Transforme un echantillon en vecteur audio  \\
		\fun{GetNote} & \bigO{1} & Transforme une hauteur en note  \\
		\hline
		\fun{Merge} & \bigO{n} &  Longueur de MusInt \\
		\fun{MergeAux} & \bigO{n} & max(longueur Vec1, longueur Vec2)  \\		
		\hline
		\fun{RepeteN} & \bigO{n} & Valeur de N \\
		\fun{RepeteD} & \bigO{n} & Valeur de Duree \\ 
		\fun{Clip} & \bigO{n} & Longueur de Vec  \\
		\fun{Echo} & \bigO{n} & Longueur de Music  \\
		\fun{CalcFirstIntensity} & \bigO{n} & Nombre de répetition  \\
		\fun{Fondu} & \bigO{n} & Longueur de Vec  \\
		\fun{FonduE} & \bigO{n} & max(longueur Vec1, longueur Vec2)  \\
		\fun{Couper} & \bigO{n} & Longueur de Vec  \\
		\fun{EnveloppeADSR} & \bigO{n} & Longueur de Vec \\
	\hline
	\end{tabular}
	\caption{Complexités des fonctions de \fun{Mix}.}
	\label{tab:complexite_mix}
\end{table}

\section{Extensions}

Afin d'améliorer le rendu sonore de notre morceau 
ainsi que d'agrandir la gamme de fonctions qui modifient les morceaux, 
nous avons réaliser plusieurs extensions.

\subsection{Enveloppe sonore trapézo\"idale}

La première est une enveloppe sonore trapézo\"idale, elle remplit une fonction de lissage.
Autrement dit, elle permet d'adoucir les bruits désagréables créés entre chaque note.
L'implémentation de cette enveloppe est relativement simple dans 
un sens où elle consiste à appliquer un fondu (fonction \fun{Fondu}) sur chaque note. 
Le fondu une augmentation linéaire du volume en début de note puis 
une diminution également linéaire du volume à la fin de la note. 
Après quelques observations et comparaisons de la durée du fondu, 
nous constatons qu'avec une durée d'un dixième de celle de la note, 
on adoucit parfaitement toutes les imperfections sans pour autant 
réduire le volume.


\subsection{Enveloppe sonore ADSR}
\label{subsec:adsr}

Le rôle joué par l'enveloppe sonore ADSR est le même que celui de l'enveloppe trapézo\"idale. 
Elle permet cependant plus de liberté : il y a 4 paramètres. 
L'\emph{attack} qui correspond au temps nécessaire pour arriver linéairement 
au volume maximum, le \emph{decay} qui indique le temps qu'il faut pour faire régresser 
le volume jusqu'à atteindre un palier, le \emph{sustain} au cours duquel le volume est constant. 
Il reste finalement le \emph{release}, qui est la durée avant la fin de la note pendant laquelle 
le volume décroit linéairement pour finalement être nul.
Cette extension est implémentée dans la fonction \fun{EnveloppeADSR} 
et nous l'utilisons dans \fun{MixVoix} (avec les paramètres qui nous semble les plus 
efficaces) pour adoucir chaque note.


\subsection{Instrument}

Cette extension permet d'utiliser des instruments pour jouer différentes notes, 
ce qui nous permet d'avoir un plus grand choix pour créer nos morceaux, 
ainsi que de jouer plusieurs instruments à la fois. 

La fonction \fun{Instrument} est implémentée dans Inteprete.oz, 
on utilise ensuite deux fonctions auxiliaires, \fun{UseWav} qui permet 
de transformer un échantillon en un vecteur audio, puis \fun{GetNote} qui permet 
de déterminer une note en fonction de la hauteur de l'échantillon. 
Les cas où la durée de l'échantillon est différente de celle du vecteur audio 
sont évidemment pris en compte.
Ceci est fait en combinant les fonctions \fun{Couper} et \fun{RepeteD}. 
Nous finissons par lisser les vecteurs audios avec une enveloppe ADSR (section \ref{subsec:adsr}).

\paragraph{Remarque}
Pour une raison inconnue, nous rencontrons un problème lors de l'exécution de \fun{Instrument}.
Nous avons donc séparé cette extension dans un fichier \textit{instrument.oz}.



\end{document}
